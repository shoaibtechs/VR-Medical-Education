\chapter{Literature overview}
\label{ch:lit-review}
This literature review explores the integration of metaversion in healthcare, highlighting its potential for innovative solutions such as medical training, telemedicine, mental health therapy, and patient empowerment. It examines historical perspectives, technological foundations, practical applications, ethical considerations, and future prospects.
\section{All one needs to know about metaverse: A complete survey on technological singularity, virtual ecosystem, and research agenda 
	\cite{lee2021all}\cite{BookChapter1}}
\textbf{Methodology:}\\The article explores the metaverse's evolution, highlighting core technologies like augmented reality, AI, and blockchain, user-centric factors like avatars, content creation, and virtual economies, and proposes a research agenda. \\
\textbf{Limitations:}\\The article offers a broad overview of metaverse evolution, but lacks in-depth analysis, discusses challenges like ethical issues and inequality, and relies on a survey-based approach without empirical support. \\
\textbf{Results:}\\Comprehensive framework for examining metaverse development.
Concrete research agenda for metaverse development
\section{Healthcare in metaverse: A survey on current metaverse applications in healthcare
	\cite{bansal2022healthcare}\cite{JournalArticle12}}
\textbf{Methodology:}\\
The article explores metaverse applications in healthcare, highlighting recent technological advancements, challenges, and potential improvements, emphasizing the need for sustainable, innovative solutions..\\
\textbf{Limitations:}\\Existing healthcare systems limitations revealed during COVID-19 pandemic. Surge in healthcare innovation using virtual environments for alternative systems. \\
\textbf{Results:}\\The metaverse for healthcare requires addressing connectivity, privacy, security, integration, interoperability, user experience, and technical issues with VR and AR technologies.
\section{Medical Metaverse: Technology, Applications, Challenges and Future
	\cite{shao2023medical}}
\textbf{Methodology}\\The paper explores the use of metaverse technologies in healthcare, identifying challenges and proposing solutions for efficient diagnosis, education, and treatment in healthcare settings.\\
\textbf{Limitations:}\\The paper provides a thorough overview of the Metaverse's potential in healthcare, highlighting its applications, technologies, and challenges, but lacks case studies and real-world implementations due to limited literature.\\
\textbf{Results:}\\ Review of technologies and applications of the metaverse. 
Exploration of potential and future direction in healthcare.
\section{The Metaverse for Healthcare: A Survey of Potential Applications, Challenges, and Future Directions.
	\cite{yendurimetaverse}\cite{JournalArticle10}}
\textbf{Methodology:}\\The paper explores the use of the Metaverse in healthcare, highlighting applications, technologies, and projects, while identifying challenges and proposing future research directions, focusing on AI, VR, AR, IoMT, robotics, and quantum computing.\\
\textbf{Limitations:}\\Adopting AI-enabled Metaverse in healthcare faces challenges like high-speed communication, massive computation, security concerns, data loss, heterogeneity of devices, and cost implications, impacting its comprehensiveness.\\
\textbf{Results:}\\
Medical education uses VR for learning body structures, improving future doctors' quality. However, AI-enabled Metaverse risks patient privacy and ethical issues, potentially leading to medical errors. Edge computing for real-time data retrieval requires additional devices, posing network scalability challenges.
\section{Revolutionizing Medical Education with Metaverse
	\cite{baskar2022revolutionizing}\cite{JournalArticle9}}
\textbf{Methodology:}\\Interdependence of characteristics in virtual teaching model.
Overcoming obstacles through technological advancements in education.\\
\textbf{Limitations:}\\The paper neglects to address technical challenges, adoption barriers, ethical considerations, and perspectives of educators, healthcare providers, and patients in integrating metaverse technology in medical education, including cost and infrastructure.\\
\textbf{Results:}\\ The paper focuses on methods for revolutionizing medical education using metaverse. It discusses the potential of metaverse in providing interactive and immersive experiences in healthcare.
\section{The Metaverse in Medical Education and Clinical Practice
	\cite{juan2023metaverse}}
\textbf{Methodology:}\\Use of immersive virtual and augmented environments.
Utilization of different models of stereoscopic vision glasses\\
\textbf{Limitations:}\\No specific limitations mentioned in the abstract. Further details on limitations not provided in the text.\\
\textbf{Results:}\\The paper presents immersive virtual and augmented environments for medical training. These environments use stereoscopic vision glasses to create a metaverse for enhanced medical training.
\section{Virtual reality and the transformation of medical education
	\cite{pottle2019virtual}}
\textbf{Methodology:}\\The paper discusses the use of simulation in clinical training but highlights its resource-intensive nature. It advocates for the adoption of virtual reality (VR) in medical education, highlighting its effectiveness in healthcare and its potential for providing quality, geographically independent training.\\
\textbf{Limitations:}\\
The paper generalizes VR's effectiveness in medical education without addressing specific limitations or challenges, lacks in-depth analysis of barriers, and overlooks technological constraints crucial for real-world implementation, presenting a one-sided view and overlooking areas for further research.\\
\textbf{Results:}\\Digital transformation has a great impact on medical education.
Inclusion of AI and VR benefits medical students.
\section{A Virtual Reality for the Digital Surgeon
	\cite{velazquez2021virtual}}
\textbf{Methodology:}\\The paper reviews VR technology's potential in healthcare, surgical education, and support tools, highlighting its potential to optimize patient data, improve surgical outcomes, and revolutionize healthcare.\\
\textbf{Limitations:}\\The paper's limited scope may limit its generalizability, overlooking challenges in implementing VR in healthcare. Additionally, reliance on a limited dataset or specific sources could affect the depth and reliability of the findings, potentially affecting the robustness of conclusions about VR's impact on surgical education and clinical practice.\\
\textbf{Results:}\\ VR has the potential to improve surgical education and skills.
VR can optimize preoperative planning and intraoperative support in clinical practice.
\section{Transforming medical education and training with VR using M.A.G.E.S.
	\cite{ProceedingsArticle}}
\textbf{Methodology:}\\The paper introduces a new VR software system for healthcare training, specifically Psychomotor Virtual Reality Surgical Training, which enhances surgeon skills through gamification and advanced interactability, and supports multiple surgeons and assistants.\\
\textbf{Limitations:}\\The paper primarily discusses orthopedic surgeries, neglecting generalizability, cost-effectiveness, VR surgical training solution validation, integrated educational curriculum effectiveness, long-term skill retention, and user experience feedback.\\
\textbf{Results:}\\The paper focuses on methods for revolutionizing medical education using metaverse. It discusses the potential of metaverse in providing interactive and immersive experiences in healthcare.
\section{Virtual Reality in Medicine
	\cite{JournalArticle}\cite{JournalArticle11}}
\textbf{Methodology}\\Multimodal interactions between user and virtual environment.
Technical requirements and design principles of input devices, displays, and rendering techniques.\\
\textbf{Limitations:}\\Physiological constraints. Technical requirements and design principles of multimodal input devices, displays, and rendering techniques.\\
\textbf{Results:}\\ Examples of virtual reality applications in surgical training, intra-operative augmentation, and rehabilitation. Provides technical requirements and design principles for virtual reality in medicine.
\section{A Virtual Environment for Training and Assessment of Surgical Teams
	\cite{papagiannakis2018virtual}}
\textbf{Methodology:}\\The paper explores the use of Collaborative Virtual Environments (CVEs) in surgical team training and assessment to enhance remote interactions and improve teamwork skills. The proposed CVE architecture supports team training and assessment in surgical simulations, reducing costs and requiring live subjects.\\
\textbf{Limitations:}\\Cost reduction for training is a limitation.
Use of guinea pigs and anatomical specimens is reduced.\\
\textbf{Results:}\\Proposed architecture for training and assessing team skills during surgery. Use of statistical models to monitor and assess team performance.
\section{Virtual reality technology and its application in modern medicine
	\cite{JournalArticle}}
\textbf{Methodology:}\\The paper discusses the use of virtual reality (VR) technology in medical applications such as surgery, telemedicine, and patient education, comparing it to traditional methods and highlighting its advantages and limitations.\\
\textbf{Limitations:}\\Limited access to VR technology in medical settings. Challenges in integrating VR into existing medical practices.\\
\textbf{Results:}\\ VR applied in virtual human, assisted diagnosis, surgery simulation. VR used in virtual telemedicine for medical purposes.
\section{Role of virtual reality for healthcare education
	\cite{BookChapter}}
\textbf{Methodology:}\\The research paper explores the use of VR technology in healthcare education, highlighting its role in storing patient data as 3D points, enhancing clinical skills, and providing flexible training for practitioners, ultimately improving the quality and standards of medical education.\\
\textbf{Limitations:}\\
Few are using VR to evaluate medical students' success. VR technology not widely used for medical training evaluation.\\
\textbf{Results:}\\ VR improves learning and training of medical practitioners.
VR enhances comprehension of anatomy and clinical outcomes.
\section{Next-Gen Mulsemedia: Virtual Reality Haptic Simulator’s Impact on Medical Practitioner for Higher Education Institutions
	\cite{journalarticle4}\cite{JournalArticle5}\cite{JournalArticle8}}
\textbf{Methodology:}\\ The study used the core motivation hypothesis to boost motivation in the classroom. The study used the attention, relevance, confidence, and satisfaction (ARCS) model to analyze the impact of virtual reality on student motivation and content update implementation.\\
\textbf{Limitations:}\\ Lack of research on consequences of virtual reality
Early stage of virtual reality technology research.\\
\textbf{Results:}\\ Virtual reality simulators improve student motivation and learning. VR has the potential to transform medical education.

\section{Design and implementation of a 3D digestive teaching system based on virtual reality technology in modern medical education
	\cite{Journal1}}
\textbf{Methodology:}\\
DX technologies: VR, AR, MR, XR, 3D images, holograms, AI. Utilization of HMDs, wearable sensors, 5G, and Wi-Fi.\\
\textbf{Limitations:}\\ Lack of systematic methodology, affecting evidence level
Heterogeneity in XR techniques studies, limiting systematic reviews.\\
\textbf{Results:}\\ The 3D digestive teaching system using VR significantly improved student understanding and retention, leading to higher test scores and increased engagement. This demonstrates VR's effectiveness in enhancing medical education.

\section{A Virtual Operating Room for Context-Relevant Training
	\cite{JournalArticle2}}
\textbf{Methodology:}\\
Virtual agents with unique personalities and knowledge structures defined. Immersive Virtual Operating Room (VOR) simulating surgical procedures described \\
\textbf{Limitation:}\\
Current medical simulators lack addressing errors in healthcare system. Existing simulators focus on procedural skills, not team dynamics.\\
\textbf{Results:}\\ Pilot session with surgical resident unfamiliar with VOR showed challenges. VOR allows team-based surgical training with virtual expert agents.

\section{Virtual reality surgical training and assessment system
	\cite{JournalArticle3}\cite{JournalArticle7}}
\textbf{Methodology:}\\ Soft-tissue model based on volumetric mass-spring system
Texture mapping for realistic organ representation and space perception.\\
\textbf{Limitation:}\\ Limits of realism in surgical simulation Challenges in obtaining reliable measures of surgical skills.\\
\textbf{Results:}\\ VR surgical system based on C-source code and OpenGL.
System handles accurate interactions between soft-tissue and surgical instruments